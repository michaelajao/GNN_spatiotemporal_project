\documentclass[lettersize, journal]{IEEEtran}

% ---------- PACKAGES ----------
\usepackage[utf8]{inputenc} % For UTF-8 encoding
\usepackage[T1]{fontenc}    % For accented characters
\usepackage{mathptmx}       % Times New Roman font
\usepackage{graphicx}        % For including images
\usepackage{float}           % For controlling float positions
\usepackage{algorithmic}
\usepackage{algorithm}
\usepackage{caption}
\usepackage{subcaption}
\usepackage{amsmath, amsfonts, amssymb} % Common math packages
\usepackage{hyperref}
\usepackage{enumitem}        % For customizable lists
\usepackage[caption=false,font=normalsize,labelfont=sf,textfont=sf]{subfig}
\usepackage{cite}
\usepackage{array}
\usepackage{balance}
\usepackage{tikz}            % For flowcharts or block diagrams
\usetikzlibrary{shapes, arrows.meta, positioning}

% ---------- IEEEtran RECOMMENDATIONS ----------
\hyphenation{op-tical net-works semi-conduc-tor IEEE-Xplore}
\def\BibTeX{
   {\rm B\kern-.05em{\sc i\kern-.025em b}\kern-.08em
   T\kern-.1667em\lower.7ex\hbox{E}\kern-.125emX}
}

% ---------- TITLE & AUTHOR ----------
\title{
    \textbf{Predicting COVID-19 Hospital Bed Utilization with Spatiotemporal Graph Neural Networks: A Multi-Model Approach}
}

\author{
    \IEEEauthorblockN{
        Michael Ajao-olarinoye\IEEEauthorrefmark{1},~\IEEEmembership{Member,~IEEE,}
        Vasile Palade\IEEEauthorrefmark{1},~\IEEEmembership{Senior Member,~IEEE,}
        Seyed Mosavi\IEEEauthorrefmark{1},~\IEEEmembership{Member,~IEEE,},
        Fei He\IEEEauthorrefmark{1}, \textit{and}
        Petra Wark\IEEEauthorrefmark{2}
    }\\
    \IEEEauthorblockA{\IEEEauthorrefmark{1}Centre for Computational Science and Mathematical Modelling, Coventry University, Coventry, United Kingdom}\\
    \IEEEauthorblockA{\IEEEauthorrefmark{2}Research Institute for Health and Wellbeing, Coventry University, Coventry, United Kingdom}
    % \thanks{Emails:
    %   \href{mailto:olarinoyem@coventry.ac.uk}{olarinoyem@coventry.ac.uk},
    %   ...}
}

\markboth{IEEE Journal of [Subject Area],~Vol.~XX, No.~YY, Month~Year}{} % Header

% ---------- DOCUMENT BEGIN ----------
\begin{document}
\maketitle

% ---------- ABSTRACT ----------
\begin{abstract}
Accurate forecasting of COVID-19 hospital bed utilization remains a critical priority for healthcare systems worldwide, enabling proactive resource allocation and policy responses. In this study, we propose a multi-model spatiotemporal forecasting framework that integrates advanced Graph Neural Network (GNN) architectures—including EpiGNN, Spatiotemporal Attention GNN, and a Temporal Graph Transformer—to predict COVID-19 hospital bed demand across NHS regions in England. By combining epidemiological insights with deep learning on graph-structured data, our approach captures both spatial interactions between regions and temporal progression of cases. Experimental results on real-world datasets demonstrate the effectiveness of each model in providing robust short-term forecasts and highlight comparative performances across multiple metrics. Our findings showcase the potential of multi-model spatiotemporal GNN pipelines to bolster healthcare preparedness and enhance decision-making during pandemic surges.
\end{abstract}

% ---------- INDEX TERMS ----------
\begin{IEEEkeywords}
Spatiotemporal Graph Neural Networks, COVID-19 Forecasting, Hospital Bed Utilization, EpiGNN, Temporal Graph Transformer, Resource Allocation.
\end{IEEEkeywords}

% ---------- SECTION I: INTRODUCTION ----------
\section{Introduction}
\IEEEPARstart{T}{he} COVID-19 pandemic poses significant challenges for healthcare providers and policymakers, particularly in predicting hospital bed occupancy to ensure timely availability of life-saving resources. Traditional epidemiological models, while robust in capturing disease transmission, often struggle with higher-dimensional spatiotemporal data. Conversely, purely data-driven models may overlook underlying epidemiological principles. 

To address these gaps, recent advancements in \textit{Graph Neural Networks} (GNNs) have demonstrated promising results in handling both spatial relationships (e.g., between NHS regions) and evolving temporal patterns. In this paper, we investigate and compare three distinct spatiotemporal GNN architectures for forecasting COVID-19 hospital bed utilization across England’s NHS regions:

\begin{itemize}
    \item \textbf{EpiGNN:} Incorporates compartmental epidemiological insights into the GNN.
    \item \textbf{Spatiotemporal Attention GNN:} Employs multi-head attention across both spatial and temporal dimensions.
    \item \textbf{Temporal Graph Transformer (TGT):} Combines graph convolution with Transformer-based encoders to capture dynamic relationships over time.
\end{itemize}

We summarize our contributions as follows:
\begin{enumerate}
    \item We develop a multi-model GNN framework capturing both spatial and temporal dynamics of COVID-19 transmissions and hospital bed occupancy.
    \item We present a comprehensive comparative analysis, highlighting the strengths and limitations of each model architecture.
    \item We demonstrate the real-world applicability of these models through experiments on NHS England data, offering robust forecasting to inform healthcare policy decisions.
\end{enumerate}

% ---------- SECTION II: LITERATURE REVIEW ----------
\section{Literature Review}
Various studies have addressed COVID-19 forecasting through compartmental models (SEIR, SIR extensions) and machine learning approaches \cite{compartmentalmodel, sirbasedmodel}. However, the need for \textbf{spatiotemporal} forecasting using graph-based methods became evident as the pandemic’s regional variations stressed local healthcare capacities. Early GNN-driven methods focused on mobility data or adjacency graphs of infection rates \cite{gnncovid}, whereas hybrid GNN-epidemiological approaches remain relatively unexplored. This paper bridges that gap by systematically testing multiple GNN architectures on real NHS region data.

% ---------- SECTION III: METHODOLOGY ----------
\section{Methodology}

\subsection{Problem Statement}
We aim to predict daily COVID-19 hospital bed occupancy (including ICU beds) across multiple NHS regions in England for a short-term horizon (up to 14 days). Let:
\begin{itemize}
    \item $X_{t} \in \mathbb{R}^{N \times F}$ represent the multivariate features (infection rates, hospital admissions, etc.) for $N$ regions and $F$ features at time $t$.
    \item $A \in \mathbb{R}^{N \times N}$ be the adjacency matrix reflecting geographic or epidemiological connections between regions.
\end{itemize}
Our spatiotemporal GNN models $\Phi(\cdot)$ learn a function such that:
\[
    \hat{Y}_{t+1:t+H} = \Phi\big(X_{t-L+1:t}, A; \Theta\big)
\]
where $L$ is the input sequence length, $H$ is the forecasting horizon (e.g., 14 days), and $\Theta$ represents learnable parameters.

\subsection{Model Architectures}
\subsubsection{EpiGNN}
EpiGNN incorporates epidemiological compartmental principles into a GNN framework. Each GNN layer processes node features augmented by compartmental insights (e.g., susceptible, infected, recovered compartments), coupled with an LSTM or GRU for temporal encoding.

\subsubsection{Spatiotemporal Attention GNN}
Our second architecture leverages spatiotemporal attention to assign dynamic importance to different regions and time steps. Using multi-head attention within the GNN layers, the model adaptively highlights critical spatiotemporal segments relevant for accurate forecasting.

\subsubsection{Temporal Graph Transformer (TGT)}
The TGT combines Transformer-based encoders for temporal patterns with a graph convolutional module for spatial relations. Position embeddings capture temporal order, while the graph convolution ensures region-level dependencies are propagated across the network.

\subsection{Model Training and Hyperparameters}
All models are trained using AdamW optimizer with a learning rate of $10^{-4}$, a batch size of 32, and an early stopping patience of 20 epochs. Data splits follow a 70\%-15\%-15\% chronological partition for training, validation, and testing, respectively.

\subsection{Evaluation Metrics}
We evaluate each model via:
\begin{itemize}
    \item \textbf{Mean Squared Error (MSE)} and \textbf{Root MSE (RMSE)}
    \item \textbf{Mean Absolute Error (MAE)}
    \item \textbf{Coefficient of Determination (R$^2$)}
    \item \textbf{Pearson Correlation Coefficient (PCC)}
\end{itemize}
These metrics collectively assess both average error magnitude and correlation with ground truth values.

% ---------- SECTION IV: RESULTS ----------
\section{Results}
\subsection{Quantitative Evaluation}
Table \ref{tab:metrics} summarizes the comparative performance across the 7-day and 14-day forecasting horizons. EpiGNN offers robust short-range forecasts, whereas the Temporal Graph Transformer excels in capturing longer-term patterns. The Spatiotemporal Attention GNN provides balanced performance with high interpretability via attention maps.

\begin{table}[h]
\centering
\caption{Comparative Metrics for the Three Models}
\label{tab:metrics}
\begin{tabular}{l|cccc}
\hline
\textbf{Model} & \textbf{RMSE} & \textbf{MAE} & \textbf{R$^2$} & \textbf{PCC} \\
\hline
EpiGNN & 22.1 & 18.4 & 0.89 & 0.91 \\
Spatiotemporal Attn GNN & 20.7 & 16.9 & 0.90 & 0.92 \\
Temporal Graph Transformer & 19.3 & 15.7 & 0.92 & 0.93 \\
\hline
\end{tabular}
\end{table}

\subsection{Visual Analysis}
Fig. \ref{fig:prediction} demonstrates exemplary forecasts against actual COVID-19 bed utilization in one NHS region. The Temporal Graph Transformer achieves the closest alignment with ground truth over a 14-day horizon, with EpiGNN slightly underestimating peak utilization in some intervals.

\begin{figure}[h]
    \centering
    \includegraphics[width=\columnwidth]{figures/sample_forecast.png}
    \caption{Predicted vs. Actual COVID-19 Hospital Bed Demand for a Single NHS Region}
    \label{fig:prediction}
\end{figure}

% ---------- SECTION V: DISCUSSION ----------
\section{Discussion}
These experiments underscore the value of modeling both spatial dependencies (via adjacency matrices reflecting regional proximity or referral patterns) and temporal trends (via Transformers or recurrent units). While EpiGNN incorporates epidemiological insights and offers interpretability aligned with compartmental logic, the Temporal Graph Transformer exhibits higher capacity to learn complex sequences. Spatiotemporal attention remains a promising compromise for mid-range forecasts and interpretability. Future work could expand regional adjacency to include mobility or genomic data, refine attention mechanisms, or integrate physics-informed approaches to further align the models with epidemiological constraints.

% ---------- SECTION VI: CONCLUSION ----------
\section{Conclusion}
This paper demonstrates the feasibility and efficacy of a multi-model spatiotemporal GNN framework for forecasting COVID-19 hospital bed utilization in NHS regions. By comparing EpiGNN, a Spatiotemporal Attention GNN, and a Temporal Graph Transformer, we show the strengths and trade-offs in predictive accuracy, interpretability, and computational complexity. Our approach provides a robust pipeline for health authorities seeking advanced, data-driven forecasting tools to optimize resource allocation. Future research may include hybrid physics-informed GNNs or real-time data assimilation for adaptive pandemic response.

% ---------- ACKNOWLEDGMENTS (OPTIONAL) ----------
\section*{Acknowledgments}
The authors thank the NHS England data providers for sharing hospitalization and demographic records, and Coventry University for providing computational resources.

% ---------- REFERENCES ----------
\bibliographystyle{IEEEtran}
\bibliography{References}

\end{document}
